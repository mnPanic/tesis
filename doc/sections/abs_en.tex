%\begin{center}
%\large \bf \runtitle
%\end{center}
%\vspace{1cm}
\chapter*{\runtitle}

\noindent \textit{Proof assistants} are programs that simplify the writing of
mathematical proofs, enabling large-scale collaboration or the formal
verification of programs. There are many proof assistants such as Mizar, Coq,
and Isabelle, based on different theories. A desirable property they may have is
described by the De Bruijn criterion: from proofs written in a high level
language, one can extract proofs in a core, easily verifiable language. This
eliminates the need to trust their implementation, as the extracted proofs can
be verified by an independent program.

This thesis presents \textit{PPA}, a proof assistant for classical first-order logic. It fulfills De Bruijn's criterion by generating proofs in the natural deduction proof system from high-level proofs, designed to resemble natural language as closely as possible. Its main proof mechanism, \texttt{by}, features an underlying heuristic solver for first-order logic, inspired by Mizar's analogous mechanism.

Some proof assistants implement \textit{existential witness extraction}. Given a
proof of $\exists \var . \pred(\var)$, they extract a *witness* \( \term \) such
that \( \pred(\var) \) holds. While this is straightforward in intuitionistic
logic due to its constructive nature, it is challenging in classical logic,
which lacks constructiveness. There are two main approaches: direct (using
semantic techniques such as classical realizability \cite{miquel-friedman}) or indirect
(via translations to another logic, such as intuitionistic).

The main contribution of this work is a practical implementation of witness
extraction. PPA achieves this indirectly by using Friedman's translation, which
allows converting classical proofs of \( \classPiTwo \) formulas of the form \(
\forall y_0 \dots \forall y_n . \exists x . \anyForm \), with the formula
$\anyForm$ having no quantifiers, into intuitionistic ones. We
describes how, once translated, these proofs can be normalized using well-known
reduction rules, which correspond to the reduction rules of lambda calculus viewed through the Curry-Howard isomorphism. Finally, a
normalized proof enables the extraction of a witness. We identify some practical
implementation details of the translation, which limit the formulas for which it
can be used (beyond the theoretical \( \classPiTwo \) restriction) and also
constrain the axiomatizations of theories that can be made.

\bigskip

\noindent\textbf{Keywords:} proof assistant, first-order logic, natural deduction, classical logic, intuitionistic logic, witness extraction, Friedman's translation.