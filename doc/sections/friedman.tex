\section{Lógica clásica}
Queremos, dado un teorema, \textit{extraer testigos de un existencial}. Por
ejemplo, si tenemos una demostración de $\exists x . p(x)$ la extracción nos
debería instanciar $x$ en un término $t$ tal que $p(t)$. Imaginemos que tenemos
el siguiente programa de PPA

\begin{verbatim}
    axiom ax: p(v)
    theorem thm: exists X . p(X)
    proof
        take X := v
        thus p(v) by ax
    end
\end{verbatim}

La demostración generada por el certificador es \textbf{clásica}. La forma más
fácil de extraer un testigo de una demostración es normalizarla y obtener el
testigo de su forma normal. Pero esto no se puede hacer en general para lógica
clásica, porque las demostraciones en general no son \textbf{constructivas}.

En la lógica clásica vale el \textit{principio del tercero excluido}, comúnmente
conocido por sus siglas en inglés, LEM (\textit{law of excluded middle}).

\begin{prop}{LEM} Para toda fórmula $\form$, es verdadera ella o su negación
    \[ \form \vee \neg \form \]
\end{prop}

Las demostraciones que usan este principio suelen dejar aspectos sin
concretizar, como muestra el siguiente ejemplo bien conocido:

\begin{theorem}\label{thm:irrat}
    Existen dos números irracionales, $a, b$ tales que $a^b$ es irracional
\end{theorem}
\begin{proof}
    Considerar el número $\sqrt{2}^{\sqrt{2}}$. Por LEM, es o bien racional o
    irracional.
    \begin{itemize}
        \item Supongamos que es racional. Como sabemos que $\sqrt{2}$ es
        irracional, podemos tomar $a=b=\sqrt{2}$.
        \item Supongamos que es irracional. Tomamos $a = \sqrt{2}^{\sqrt{2}}, b
        = \sqrt{2}$. Ambos son irracionales, y tenemos

        \[
            a^b
            = \Bigg( \sqrt{2}^{\sqrt{2}} \Big)^{\sqrt{2}}
            = \sqrt{2}^{\sqrt{2} \cdot \sqrt{2}}
            = \sqrt{2}^{2}
            = 2.
        \]

        que es racional.
    \end{itemize}
\end{proof}

La prueba no nos da forma de saber cuales son $a$ y $b$. Es por eso que en
general, tener una demostración de un teorema que afirma la existencia de un
objeto que cumpla cierta propiedad, no necesariamente nos da una forma de
encontrar tal objeto. Entonces tampoco vamos a poder extraer un testigo.

En el caso de \ref{thm:irrat}, lo demostramos de una forma no constructiva pero
existen formas constructivas de hacerlo \todo{citar}. En cambio, si consideramos
la fórmula

\[
    \exists x ((x = 1 \wedge C) \vee (x = 0 \wedge \neg C))
\]

pensando $C$ como algo indecidible, por ejemplo HALT, trivialmente podemos
demostrarlo de forma no constructiva (LEM con $C \vee \neg C$) pero no de forma constructiva.

\section{Lógica intuicionista}

Para solucionar estos problemas existe la lógica \textbf{intuicionista}, que se
puede definir como la lógica clásica sin LEM. Al ser siempre constructiva, sirve
para tener interpretaciones computacionales (como la \textit{BHK}). En esta
lógica, podemos reducir la prueba hacia una forma normal con un proceso análogo
a una reducción de cálculo $\lambda$. Luego en la forma normal se esperaría que
toda demostración de un $\exists$ sea mediante $\existsI$, explicitando el
testigo.



\cite{miquel-friedman}

\cite{selinger-friedman}