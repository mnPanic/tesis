Puntos a abordar

- mencionar realizabilidad clásica. Related work capaz en la conclusión
- hacer una investigación de otras formas de hacer witness extraction. Capaz no es original lo nuestro (y capaz Coq lo banca con realizabilidad).

\begin{itemize}
    \item Motivación, limitaciones de lógica clásica. Demostración sqrt 2
    \item Lógica intuicionista
    \item Como necesitamos reducir en ND, necesitamos la demo en ND. Escribirla
    en este caso.
    \item También queremos para $\piZeroTwo$, mostrar la extensión en ND.
    \item En realidad no nos sirve $\transDNeg{\Gamma}$, queremos dejarlo como
    está y demostrar que los axiomas demuestran sus traducciones. Pero no vale
    siempre (buscar c.ej), caracterizar cuando.
    \item Sumarizar cómo queda, vincular con reducción. Mostrar ejemplos en PPA
    que funcionan y ejemplos que no.
    \item Extensión a demostraciones. Mostrar algunos ejemplos interesantes (y
    los que usen los lemas dNegRElim y rElim)
    \item Lemas para demostraciones: dNegRElim (relacionar con \ref{ppa-cert:sec:abs-reasoning}), rElim, tNegRElim
    \item Reducción (buena explicación
    \url{https://plato.stanford.edu/entries/natural-deduction/}). En realidad se
    conoce como \textbf{normalization}.
    \begin{itemize}
        \item Similitud con reducción en cálculo lambda.
        \item Ejemplos de LP y todo LPO
        \item substHyp, substVar en proofs
        \item Argumentos de que es correcto y completo?
        \item Small step vs big step
    \end{itemize}
\end{itemize}

\newpage

En los capítulos anteriores vimos como el lenguaje PPA puede ser usado para
escribir demostraciones de alto nivel, que son certificadas generando
demostraciones de bajo nivel usando el sistema lógico de deducción natural.
Ahora vamos a introducir una nueva funcionalidad: la \textbf{extracción de testigos}.

\begin{multicols}{2}
    \begin{figure}[H]
        \lstinputlisting{listings/extract/exists.ppa}
        \caption{Extracción simple}
        \label{fri:prog:exists}
    \end{figure}

    \begin{figure}[H]
        \lstinputlisting[firstline=2]{listings/extract/indirect.ppa}
        \caption{Extracción indirecta}
        \label{fri:prog:indirect}
    \end{figure}
\end{multicols}

Por ejemplo, en el programa \fullref{fri:prog:exists} la extracción nos permitirá encontrar un término $t$ que sea testigo de $\exists
x. p(x)$, es decir que cumpla $p(t)$. En este caso es fácil encontrarlo a ojo
sobre la demostración de PPA, sería \lstinline{v}. Pero puede haber casos en
donde no sea tan trivial, como \fullref{fri:prog:indirect}, en donde la
instanciación de la variable en un término es indirecta.

Buscamos un mecanismo
general, que nos permita para cualquier demostración de una fo

\todo{Hablar sobre y citar classical realizability, no tengo ni idea.}


Queremos, dado un teorema, \textit{extraer testigos de un existencial}. Por
ejemplo, si tenemos una demostración de $\exists x . p(x)$ la extracción nos
debería instanciar $x$ en un término $t$ tal que $p(t)$. Imaginemos que tenemos
el siguiente programa de PPA

\begin{verbatim}
    axiom ax: p(v)
    theorem thm: exists X . p(X)
    proof
        take X := v
        thus p(v) by ax
    end
\end{verbatim}

¿Cómo hacemos para extraer La demostración generada por el certificador es \textbf{clásica}. La forma más
fácil de extraer un testigo de una demostración es normalizarla y obtener el
testigo de su forma normal. Pero esto no se puede hacer en general para lógica
clásica, porque las demostraciones en general no son \textbf{constructivas}.

En la lógica clásica vale el \textit{principio del tercero excluido}, comúnmente
conocido por sus siglas en inglés, LEM (\textit{law of excluded middle}).

\begin{prop}[LEM] Para toda fórmula $\form$, es verdadera ella o su negación
    \[ \form \vee \neg \form \]
\end{prop}

Las demostraciones que usan este principio suelen dejar aspectos sin
concretizar, como muestra el siguiente ejemplo bien conocido:

\begin{theorem}\label{fri:thm:irrat}
    Existen dos números irracionales, $a, b$ tales que $a^b$ es irracional
\end{theorem}
\begin{proof}
    Considerar el número $\sqrt{2}^{\sqrt{2}}$. Por LEM, es o bien racional o
    irracional.
    \begin{itemize}
        \item Supongamos que es racional. Como sabemos que $\sqrt{2}$ es
        irracional, podemos tomar $a=b=\sqrt{2}$.
        \item Supongamos que es irracional. Tomamos $a = \sqrt{2}^{\sqrt{2}}, b
        = \sqrt{2}$. Ambos son irracionales, y tenemos

        \[
            a^b
            = \left( \sqrt{2}^{\sqrt{2}} \right)^{\sqrt{2}}
            = \sqrt{2}^{\sqrt{2} \cdot \sqrt{2}}
            = \sqrt{2}^{2}
            = 2,
        \]

        que es racional.
    \end{itemize}
\end{proof}

Como se puede ver, la prueba no nos da forma de saber cuales son $a$ y $b$. Es
por eso que en general, tener una demostración de un teorema que afirma la
existencia de un objeto que cumpla cierta propiedad, no necesariamente nos da
una forma de encontrar tal objeto. Entonces tampoco vamos a poder extraer un
testigo.

En el caso de \fullref{thm:irrat}, lo demostramos de una forma no constructiva pero
existen formas constructivas de hacerlo \todo{citar}. Pero hay casos en donde
no. Por ejemplo, si consideramos la fórmula

\[
    \exists x ((x = 1 \wedge C) \vee (x = 0 \wedge \neg C))
\]

y pensamos en $C$ como algo indecidible, por ejemplo \texttt{HALT}, trivialmente
podemos demostrarlo de forma no constructiva (LEM con $C \vee \neg C$) pero no
de forma constructiva.

\section{Lógica intuicionista}

Para solucionar estos problemas existe la lógica \textbf{intuicionista}, que se
puede definir como la lógica clásica sin LEM. Al no contar con ese principio,
las demostraciones son constructiva. Esto permite por un lado para tener
interpretaciones computacionales (como la \textit{BHK}) y además que exista la
noción de \textit{forma normal} de uan demostración. Existen métodos bien
conocidos para reducir prueba hacia su
forma normal con un proceso análogo a una reducción de cálculo $\lambda$. Luego
en la forma normal se esperaría que toda demostración de un $\exists$ sea
mediante \ruleExistsI, explicitando el testigo.

Al no tener LEM, tampoco valen principios equivalentes, como la eliminación de
la doble negación \todo{hablar un poco más de esto}

\section{Traducción de Friedman}

\subsection{Traducción de doble negación}

Queremos extraer testigos de las demostraciones generadas por el certificador de
PPA, pero son en lógica clásica. Sabemos que podemos hacerlo para lógica
intuicionista. ¿Cómo conciliamos ambos mundos?

Existen muchos métodos que permiten embeber la lógica clásica en la
intuicionista \todo{citar}. Un mecanismo general es la traducción de
\textbf{doble negación}, que intuitivamente consiste en agregar una doble
negación recursivamente a toda la fórmula. Por ejemplo

\newcommand{\gN}[1]{#1^{\text{N}}}

\begin{definition}[Traducción \textit{Gödel-Gentzen}] Dada una fórmula $\form$ se asocia con otra $\gN{\form}$. La
    traducción se define inductivamente en la estructura de la fórmula de la
    siguiente forma

    \begin{align*}
        \gN{\bot} &= \bot\\
        \gN{\form} &= \neg\neg \form \quad \text{con $\form \neq \bot$ atómica}\\
        \gN{(\form \wedge \formTwo)} &= \gN{\form} \wedge \gN{\formTwo}\\
        \gN{(\form \vee \formTwo)} &= \neg(\neg\gN{\form} \wedge \neg\gN{\formTwo})\\
        \gN{(\form \rightarrow \formTwo)} &= \gN{\form} \rightarrow \gN{\formTwo}\\
        \gN{(\forall \var . \form)} &= \forall \var . \gN{\form}\\
        \gN{(\exists \var . \form)} &= \neg \forall \var . \neg \gN{\form}
    \end{align*}
\end{definition}

\begin{theorem}
    Si tenemos $\judgC \form$, luego $\judgI \gN{\form}$
\end{theorem}

Esto significa que dada una demostración en lógica clásica, podemos obtener una
en lógica intuicionista de su traducción. Pero esto no es exactamente lo que
queremos, porque por ejemplo

\[
    \gN{(\exists x. p(x))} = \neg \forall x . \neg\neg\neg p(x)
\]

Que al no ser una demostración de un $\exists$, al reducirla no necesariamente
obtendremos un testigo.

\subsection{El truco de Friedman}

La idea de Friedman \cite{miquel-friedman} es generalizar la traducción
Gödel-Gentzen reemplazando la negación intuicionista $\neg \form \equiv A
\rightarrow \bot$ por una relativa $\negR \form \equiv \form \rightarrow R$ que
está parametrizada por una fórmula arbitraria $R$. Esto nos va a permitir, con
una elección particular de $R$,
traducir una demostración clásica de una fórmula $\sigmaZeroOne$ (e incluso
$\piZeroTwo$) a una intuicionista, y usarla para demostrar \textbf{la fórmula
original}. Finalmente podremos reducirla y hacer la extracción de forma usual.

\begin{definition}[Traducción de doble negación relativizada]
    \begin{align*}
        \transDNeg{\bot} &= \bot\\
        \transDNeg{\form} &= \negR\negR \form \quad \text{con $\form \neq \bot$ atómica}\\
        \transDNeg{(\form \wedge \formTwo)} &= \transDNeg{\form} \wedge \transDNeg{\formTwo}\\
        \transDNeg{(\form \vee \formTwo)} &= \negR(\negR\transDNeg{\form} \wedge \negR\transDNeg{\formTwo})\\
        \transDNeg{(\form \rightarrow \formTwo)} &= \transDNeg{\form} \rightarrow \transDNeg{\formTwo}\\
        \transDNeg{(\forall x . \form)} &= \forall x . \transDNeg{\form}\\
        \transDNeg{(\exists x . \form)} &= \negR \forall x . \negR \transDNeg{\form}
    \end{align*}
\end{definition}

\begin{theorem}
    Si $\ctx \judgC \form$, luego $\transDNeg{\ctx} \judgI \transDNeg{\form}$

    Veremos esta extensión de la traducción a contextos y demostraciones más adelante.
\end{theorem}

Veamos cómo podemos usarla para, dada una demostración clásica de $\exists \var
\form$ obtener una intuicionista.

\begin{prop}
    Sea $\someProof$ una demostración clásica de $\exists \var . \form$, y
    $\form$ una fórmula atómica.
    Si tenemos
    \[
        \ctx \judgC \exists \var . \form,
    \]
    luego
    \[
        \transDNeg{\ctx} \judgI \exists \var . \form.
    \]
\end{prop}
\begin{proof}

Aplicando la traducción, tenemos que

\begin{prooftree}
    \AxiomC{$\someProof$}
    \UnaryInfC{$\ctx \judgC \exists \var . \form$}
\end{prooftree}

se traduce a

\begin{prooftree}
    \AxiomC{$\transDNeg{\someProof}$}
    \UnaryInfC{$\transDNeg{\ctx} \judgI \negR \forall \var . \negR \negR \negR \form$}
\end{prooftree}

luego, tomando $R$ como la fórmula que queremos probar, $\exists \var . \form$

\begin{align*}
    \transDNeg{\someProof} \vartriangleright\ & \transDNeg{\ctx} \judgI \negR \forall \var . \negR \negR \negR \form\\
    \iff & \transDNeg{\ctx} \judgI \negR \forall \var . \negR \form
    &&(\fullref{lemma:tnegr-elim})\\
    =\ &\transDNeg{\ctx} \judgI (\forall \var . (\form \rightarrow R)) \rightarrow R\\
    =\ &\transDNeg{\ctx} \judgI (\forall \var . (\form \rightarrow \exists \var . \form)) \rightarrow \exists \var . \form && (R = \exists \var \form)\\
    \Rightarrow\ &\transDNeg{\ctx} \judgI \exists \var . \form && (\fullref{obs:forall-exists})
\end{align*}
\end{proof}

\begin{lemma}\label{fri:lemma:tnegr-elim}
    $\negR\negR\negR \form \iff \negR \form$
\end{lemma}
\begin{proof}
    \todo{En deducción natural}
\end{proof}

\begin{obs}\label{fri:obs:forall-exists}
    $\judgI \forall \var (\form \rightarrow \exists \var \form)$
\end{obs}

\todo{IDem pero en ND, y también para $\forall$}
