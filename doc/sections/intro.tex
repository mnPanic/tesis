\section{Asistentes de demostraciones}

\begin{itemize}
    \item Son programs que asisten al usuario a la hora de escribir una
    demostracion. Permiten representarlas en un programa.
    \item Aplicaciones: formalización de teoremas, verificación formal de
    programas, etc.
    \item Ejemplos: Coq, isabelle (isar), Mizar
    \item Reseña histórica de Mizar
    \item Ventajas: colaboración a gran escala (confianza en el checker),
    chequear el output de los LLMs
\end{itemize}

\subsection{Mizar}

\todo{Mizar}

\section{Arquitectura de PPA}

\begin{itemize}
    \item Por qué certificados (criterio de De Brujin)
    \item Certificados están en deducción natural. Sistema lógico que permite
    construir demostraciones mediante reglas de inferencia.
    \item PPA es un lenguaje que genera demostraciones "de bajo nivel" ND.
    \item Implementado en Haskell
\end{itemize}

\section{Lógica de primer órden}

\todo{Definiciones, repaso, lo necesario}