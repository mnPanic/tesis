En la sección pasada vimos cómo se puede usar PPA para demostrar teoremas. Pero,
¿cómo funciona por detrás? ¿Como chequea la validez lógica de las
demostraciones?

\section{Certificados}

Los programas de PPA se \textbf{certifican}, generando una demostración en
deducción natural. Esto hace que la herramienta cumpla con el \textit{criterio
de De Brujin}. Generando u

\section{Certificador}


\todo{Cut como meta teorema}

\todo{DnegElim como meta teorema, y cómo permite razonar por el absurdo}

\subsection{Unificación}
\label{ppa:sec:unification}