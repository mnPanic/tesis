%\begin{center}
%\large \bf \runtitulo
%\end{center}
%\vspace{1cm}
\chapter*{\runtitulo}

\noindent Los \textit{asistentes de demostración} son programas que simplifican
la escritura de demostraciones matemáticas, permitiendo la colaboración masiva o
la verificación formal de programas. Existen muchos asistentes como Mizar, Coq e
Isabelle, basados en distintas teorías. Un criterio deseable que pueden cumplir
es el de De Bruijn: a partir de demostraciones en un lenguaje de alto nivel se pueden extraer demostraciones en un lenguaje núcleo fácilmente verificable. Esto elimina la necesidad de tener que confiar en la implementación del asistente, ya que las demostraciones pueden ser verificadas por un programa independiente.

En esta tesis se presenta PPA, un asistente de demostración para lógica clásica
de primer orden. Cumple con el criterio de De Bruijn, generando demostraciones
en el sistema lógico de \textit{deducción natural} a partir de programas
escritos en un lenguaje de alto nivel, cuyo objetivo es ser similar a cómo
serían en lenguaje natural. Tiene un mecanismo principal de demostración, el \texttt{by},
que por debajo cuenta con un \textit{solver} heurístico para lógica de primer
orden que facilita la escritura de demostraciones. Está inspirado en el
mecanismo análogo en Mizar.

Algunos asistentes implementan la \textit{extracción de testigos de
existencial}. Dada una demostración de $\exists \var . \pred(\var)$, se extrae
un \textit{testigo} $\term$ tal que cumpla $\pred(\term)$. Es sencillo hacerlo sobre lógica intuicionista por su naturaleza constructiva, pero un desafío sobre lógica clásica, que no lo es. Para ello hay dos grandes categorías: directas (mediante técnicas semánticas como realizabilidad clásica \cite{miquel-friedman}) o indirectas (mediante traducciones a otra lógica, como intuicionista).

El aporte principal del trabajo es una implementación práctica de la extracción
de testigos. PPA la implementa de forma indirecta usando la traducción de
Friedman, que permite convertir  demostraciones clásicas de fórmulas
$\classPiTwo$ de la forma $\forall \varTwo_0 \dots \forall \varTwo_n . \exists
\var . \anyForm$ a demostraciones intuicionistas. Se describe cómo una vez traducidas pueden
ser normalizadas usando reglas de reducción bien conocidas, que se corresponden con las reglas de reducción del cálculo lambda vistas desde el isomorfismo
Curry-Howard. Finalmente, de una demostración normalizada se podrá extraer un
testigo. Identificamos algunos detalles en la implementación práctica de la
traducción, que limitan las fórmulas para las cuales se puede usar (más allá de
la limitación teórica de $\classPiTwo$) y también limitan las axiomatizaciones
de las teorías que se pueden hacer.

\bigskip

\noindent\textbf{Palabras claves:} asistente de demostración, lógica de primer orden, deducción natural, lógica clásica, lógica intuicionista, extracción de testigos, traducción de Friedman.