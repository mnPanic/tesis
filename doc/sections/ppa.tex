El lenguaje PPA (\textit{Pani's Proof Assistant}) se construye sobre deducción natural. Es un asistente de demostración que permite
escribir de una forma práctica demostraciones de cualquier teoría de lógica
clásica de primer orden. En este capítulo nos vamos a centrar en el lenguaje
\ppaLang{}, implementado por \ppaTool{}, con un enfoque de referencia del
lenguaje. Los detalles teóricos de cómo está implementado se abordan en el
\namedref{chap:ppa-certifier}, y los de implementación y su instalación en el
\namedref{chap:ppa-tool}. Para introducirlo veamos un ejemplo: en la
\namedref{ppa:prog:alumnos} representamos el mismo de alumnos del
\namedref{nd:ex:exam-nd-lpo} pero con un poco más de sofisticación.

\begin{figure}
    \lstinputlisting[firstline=12,lastline=61,language=PPA,name=Alumnos]{listings/interfaz/alumnos.ppa}
    \caption{Programa de ejemplo completo en PPA. Demostraciones de alumnos y parciales.}
    \label{ppa:prog:alumnos}
\end{figure}

Vayamos parte por parte. La primera de todo programa en PPA es definir
los axiomas de la \textit{teoría de primer orden} con la que se está trabajando.
Como no se chequean tipos, no es necesario definir explícitamente los símbolos
de predicados y de función. Pero se pueden agregar a modo informativo como un
comentario. Los axiomas son fórmulas que siempre son consideradas como válidas.
Definimos,
\begin{itemize}
    \item \lstinline{axiom reprueba_recu_parcial_recursa}: si un alumno reprueba el
    parcial y el recuperatorio de una materia, la recursa.
    \item \lstinline{axiom reprobo_rinde}: si un alumno reprobó un examen, es porque lo
    rindió.
    \item \lstinline{axiom rinde_recu_reprobo_parcial}: si un alumno rinde el recu de
    un parcial, es porque reprobó la primer instancia.
    \item \lstinline{axiom falta_reprueba}: si un alumno falta a un examen, lo reprueba.
\end{itemize}
\lstset{firstnumber=last}
\lstinputlisting[firstline=1,lastline=23,name=Alumnos]{listings/interfaz/alumnos.ppa}

En base a eso demostramos dos teoremas. El primero, \lstinline{theorem reprueba_recu_recursa}, nos permite concluir que un alumno recursa solo a partir
de que reprueba el recuperatorio. Con el resto de los axiomas, podemos deducir
que también reprobó el parcial: si reprueba el recuperatorio es porque lo
rindió, y si rindió el recuperatorio, es porque reprobó el parcial.

\begin{figure}[H]
    \lstinputlisting[firstnumber=last,firstline=25,lastline=46,name=Alumnos]{listings/interfaz/alumnos.ppa}
\end{figure}

Para demostrar un teorema, tenemos que agotar su \textit{tesis} reduciéndola
sucesivamente con \textit{proof steps}. Una demostración es correcta si todos
los pasos son lógicamente correctos, y luego de ejecutarlos todos, la tesis se
reduce por completo.

\begin{itemize}
    \item \cmdLet{} permite demostrar un \lstinline{forall}, asignando un nombre general a la variable, y \textit{reduce} la tesis a su fórmula.
    \item \cmdSuppose{} permite demostrar una implicación. Agrega como
    hipótesis al contexto el antecedente, permitiendo nombrarlo, y reduce la
    tesis al consecuente.
    \item \cmdClaim{} inserta una sub-demostración auxiliar, cuya fórmula se
    agrega como hipótesis. No reduce la tesis.
    \item \cmdHave{} agrega una hipótesis auxiliar, sin reducir la tesis.
    \item \cmdBy{} es el mecanismo principal de demostración. Permite deducir
    fórmulas a partir de otras. Es completo para lógica proposicional, y
    heurístico para primer orden. Unifica las variables de los \lstinline{forall}.
    \item \cmdThus{} permite reducir parte o la totalidad de la tesis.
    \item \cmdHence{} es igual a thus, pero incluye implícitamente la
    hipótesis anterior a las justificaciones del \cmdBy{}.
\end{itemize}

Finalmente, a partir del teorema anterior y \lstinline{axiom falta_reprueba}
podemos demostrar que si un alumno falta a un recuperatorio, recursa la materia.

\begin{figure}[H]
    \lstinputlisting[firstnumber=last,firstline=48,lastline=61,name=Alumnos]{listings/interfaz/alumnos.ppa}
\end{figure}

\lstset{firstnumber=1}

Al ejecutarlo con \texttt{ppa}, se \textit{certifica} la demostración, generando
un certificado de deducción natural, y luego se chequea que sea correcto. Si se
escribió una demostración que no es lógicamente válida, el certificador reporta
el error. No debería fallar nunca el chequeo sobre el certificado.

\section{Interfaz}

PPA es un lenguaje que permite escribir demostraciones de cualquier teoría de
lógica de primer orden. Su sintaxis busca ser lo más parecida posible al
lenguaje natural, inspirada en Mizar y el \textit{mathematical vernacular} de Freek Wiedijk \cite{freek-mv}. También se puede entender como una notación de
deducción natural en el estilo de Fitch \cite{sep-natural-deduction}, en donde
las demostraciones son esencialmente listas de fórmulas, con las que aparecen
más adelante siendo consecuencia de las que aparecieron antes. El estilo de
Fitch y el de Gentzen son equivalentes, describen la misma relación de
derivabilidad $\ctx \judG \anyForm$. En esta sección nos
concentramos en la interfaz de usuario, sin entrar en detalle en cómo está
implementada.

Un programa de PPA consiste en una lista de \textbf{declaraciones}: axiomas y
teoremas, que se leen en orden el inicio hasta el final.

\begin{itemize}
    \item Los axiomas se asumen válidos, se usan para modelar la \textit{teoría
    de primer orden} sobre la cual hacer demostraciones

    \begin{lstlisting}[numbers=none]
axiom <name> : <form>
    \end{lstlisting}

    \item Los teoremas deben ser demostrados. En ella pueden citar
    todas las hipótesis definidas previamente.

    \begin{lstlisting}[numbers=none]
theorem <name> : <form>
proof
    <steps>
end
    \end{lstlisting}
\end{itemize}

\subsection{Identificadores}

Los identificadores se dividen en tres tipos:

\begin{itemize}
    \item \textbf{Variables} (\lstinline{<var>}).
    Son una secuencia de uno o más símbolos, que pueden ser alfanuméricos o  \verb/"-"/, \verb/"_"/. Deben comenzar por \verb/"_"/ o una letra
    mayúscula, y pueden estar seguidas de cero o más apóstrofes
    \verb/"'"/.

    También pueden ser descritas
    por la siguiente expresión regular del estilo PCRE (\textit{Perl Compatible
    Regular Expressions}):
    \begin{center}
        \verb/(\_|[A-Z])[a-zA-Z0-9\_\-]*(\')*/
    \end{center}
    \item \textbf{Identificadores} (\lstinline{<id>}).
    Son una secuencia de uno o más simbolos, que pueden ser alfanuméricos o 
    \verb/"_"/, \verb/"-"/, \verb/"?"/, \verb/"!"/, \verb/"#"/, \verb/"$"/,
    \verb/"%"/, \verb/"*"/, \verb/"+"/, \verb/"<"/, \verb/">"/, \verb/"="/, \verb/"?"/, \verb/"@"/, \verb/"^"/. Pueden estar seguidas de cero o más apóstrofes
    \verb/"'"/.

    También pueden ser
    descritas por la siguiente expresión regular:
    \begin{center}
        \verb/[a-zA-Z0-9\_\-\?!#\$\%\*\+\<\>\=\?\@\^]+(\')*/
    \end{center}
    \item \textbf{Nombres} (\lstinline{<name>}): pueden ser identificadores, o
    \textit{strings} arbitrarios encerrados por comillas dobles
    (\texttt{"..."}), que son descritos por la siguiente expresión regular
    \begin{center}
        \verb/\"[^\"]*\"/
    \end{center}
\end{itemize}

\subsection{Comentarios}

Se pueden escribir comentarios de una sola línea (\lstinline{// ...}) o multilínea
(\lstinline{/*  ... */})

\subsection{Fórmulas}

Las fórmulas están compuestas por,

\begin{itemize}
    \item \textbf{Términos}
    \begin{itemize}
        \item \textbf{Variables}: \lstinline{<var>}. Ejemplos: \lstinline{_x},
        \lstinline{X}, \lstinline{X'''}, \lstinline{Alumno}.
        \item \textbf{Funciones}: \lstinline{<id>(<term>, ..., <term>)}. Los
        argumentos son opcionales, pudiendo tener funciones 0-arias
        (constantes). Ejemplos:  \lstinline{c}, \lstinline{f(_x, c, X)}.
    \end{itemize}
    \item \textbf{Predicados}: \lstinline{<id>(<term>, ..., <term>)}. Los
    argumentos son opcionales, pudiendo tener predicados 0-arios (variables
    proposicionales). Ejemplos: \lstinline{p(c, f(w), X)}, \lstinline{A},
    \lstinline{<(n, m)}
    \item \textbf{Conectivos binarios}.
    \begin{itemize}
        \item \lstinline{<form> & <form>} (conjunción)
        \item \lstinline{<form> | <form>} (disyunción)
        \item \lstinline{<form> -> <form>} (implicación)
    \end{itemize}
    \item \textbf{Negación} \lstinline{~<form>}
    \item \textbf{Cuantificadores}
    \begin{itemize}
        \item \lstinline{exists <var> . <form>}
        \item \lstinline{forall <var> . <form>}
    \end{itemize}
    \item \lstinline{true}, \lstinline{false}
    \item \textbf{Paréntesis}: \lstinline{(<form>)}
\end{itemize}


\section{Demostraciones}

Las demostraciones consisten de una lista de \textit{proof steps} o comandos que
pueden reducir sucesivamente la \textit{tesis} (fórmula a demostrar) hasta
agotarla por completo. Corresponden aproximadamente a reglas de inferencia de
deducción natural.

\subsection{Contexto}

Las demostraciones llevan asociado un \textit{contexto} con todas las hipótesis
que fueron asumidas (como los axiomas), o demostradas: tanto teoremas anteriores
como sub-teoremas y otros comandos que agregan hipótesis a él.

\subsection{by - el mecanismo principal de demostración}

El mecanismo principal de demostración directa o \textit{modus ponens} es el
\lstinline{by}, que afirma que un hecho es consecuencia de una lista de
hipótesis (su ``justificación''). Esto permite eliminar universales e implicaciones. Por detrás hay un
\textit{solver} completo para lógica proposicional pero heurístico para primer
orden (elimina todos los cuantificadores universales de a lo sumo una hipótesis, no intenta con más
de una). Exploramos las limitaciones en \fullref{ppa-cert:sec:expressiveness}.

En general, \lstinline{<form> by <justification>} se interpreta como que
\lstinline{<form>} es una consecuencia lógica de las fórmulas que corresponden a
las hipótesis de \lstinline{<justification>}, que deben estar declaradas
anteriormente y ser parte del contexto. Ya sea por axiomas o teoremas, u otros
comandos que agreguen hipótesis (como \lstinline{suppose}). Puede usarse de dos
formas principales, \lstinline{thus} y \lstinline{have}.

\begin{itemize}
    \item \lstinline{thus <form> by <justification>}.
    
    Si \lstinline{<form>} es \textit{parte} de la tesis (ver
    \fullref{ppa:sec:and-intro}), y el \textit{solver} heurístico puede
    demostrar que es consecuencia lógica de las justificaciones, lo demuestra automáticamente y lo descarga de la tesis.

    Por ejemplo, para eliminación de implicaciones

    \lstinputlisting{listings/interfaz/by-imp.ppa}

    Y para eliminación de cuantificadores universales

    \lstinputlisting{listings/interfaz/by-forall.ppa}

    \item \lstinline{have <form> by <justification>}.
    
    Igual a \lstinline{thus}, pero permite introducir afirmaciones
    \textit{auxiliares} que no son parte de la tesis, sin reducirla, y las
    agrega a las hipótesis del contexto para su uso posterior. Por ejemplo, la demostración
    anterior la hicimos en un solo paso con el \lstinline{thus}, pero podríamos
    haberla hecho en más de uno con una afirmación auxiliar intermedia.

    \lstinputlisting{listings/interfaz/by-imp-have.ppa}
\end{itemize}

Ambas tienen su contraparte con \textit{azúcar sintáctico} que agrega
automáticamente la hipótesis anterior a la justificación, a la que también se
puede referir con guión medio (\lstinline{-}).

\begin{table}[H]
    \centering
\begin{tabular}{l|l|l}
Comando             & Alternativo             & ¿Reduce la tesis? \\
\hline
\lstinline|thus|    & \lstinline|hence|       & Sí               \\
\lstinline|have|    & \lstinline|then|        & No              
\end{tabular}
\end{table}

Por ejemplo, 
\begin{multicols}{2}
    \lstinputlisting[
        firstline=1,
        lastline=9,
    ]{listings/interfaz/by-imp-then.ppa}
    \vfill\null
    \columnbreak
    \lstinputlisting[
        firstline=11,
        lastline=23,
        firstnumber=last
    ]{listings/interfaz/by-imp-then.ppa}
\end{multicols}


En todos, el \lstinline{by} es opcional. En caso de no especificarlo en
\lstinline{thus} o \lstinline{have}, la fórmula debe ser demostrable por el
\textit{solver} heurístico sin partir de ninguna hipótesis, lo cual en
particular vale para todas las tautologías proposicionales. Por ejemplo,

\lstinputlisting{listings/interfaz/by-taut.ppa}

\subsection{Comandos y reglas de inferencia}


Muchas reglas de inferencia de deducción natural (\ref{nd:inference-rules})
tienen una correspondencia directa con comandos. Como se puede ver en
\fullref{ppa:tab:inference-rules-to-commands}, la mayor parte del trabajo
manual, detallista y de bajo nivel de escribir demostraciones en deducción
natural resuelve automáticamente con el uso \lstinline{by}.
\begin{table}[H]
    \centering
    \begin{tabular}{c|c}
    Regla & Comando \\
    \hline
    \ruleExistsI    &   \lstinline|take| \\
    \ruleExistsE    &   \lstinline|consider| \\
    \ruleForallI    &   \lstinline|let| \\
    \ruleForallE    &   \lstinline|by| \\
    \ruleOrIOne     &   \lstinline|by| \\
    \ruleOrITwo     &   \lstinline|by| \\
    \ruleOrE        &   \lstinline|cases| \\
    \ruleAndI       &   \lstinline|by| \\
    \ruleAndEOne    &   \lstinline|by| \\
    \ruleAndETwo    &   \lstinline|by| \\
    \ruleImpI       &   \lstinline|suppose| \\
    \ruleImpE       &   \lstinline|by| \\
    \ruleNotI       &   \lstinline|suppose| \\
    \ruleNotE       &   \lstinline|by| \\
    \ruleTrueI      &   \lstinline|by| \\
    \ruleFalseE     &   \lstinline|by| \\
    \ruleLEM        &   \lstinline|by| \\
    \ruleAx         &   \lstinline|by|
    \end{tabular}
    \caption{Reglas de inferencia y comandos}
    \label{ppa:tab:inference-rules-to-commands}
\end{table}


\begin{itemize}
    \item \lstinline{suppose} (\ruleImpI / \ruleNotI)
    
    Si la tesis es una implicación $\form \fImp \formTwo$, agrega el antecedente
    $\form$ como hipótesis con el nombre dado y reduce la tesis al consecuente
    $\formTwo$. Viendo la negación como una implicación $\fNot \form \equiv
    \form \fImp \fFalse$, se puede usar para introducir negaciones, tomando
    $\formTwo = \fFalse$.

    \begin{multicols}{2}
        \lstinputlisting[firstline=1, lastline=7]{listings/interfaz/suppose.ppa}
        \vfill\null
        \columnbreak
        \lstinputlisting[firstline=9, lastline=15]{listings/interfaz/suppose.ppa}    
    \end{multicols}
    \item \lstinline{cases} (\ruleOrE)
    
    Permite razonar por casos. Para cada uno se debe demostrar la tesis en su
    totalidad por separado.

    \lstinputlisting[firstline=1, lastline=11]{listings/interfaz/cases.ppa}

    No es necesario que los casos sean exactamente iguales a como están
    presentados en las hipótesis, solo debe valer que la disyunción de ellos sea
    consecuencia de ella. Es decir, para poder usar

    \begin{lstlisting}[numbers=none]
cases by h1, ..., hn
    case c1
    ...
    case cm
end
    \end{lstlisting}

    Tiene que valer \lstinline{c1 | ... | cm by h1, ..., hn}.
    
    Por lo que en el ejemplo anterior, podríamos haber usado el mismo case
    incluso si la hipótesis fuera \lstinline{~((~a | ~b) & (~c | ~a))}, pues es
    equivalente a \lstinline{(a & b) | (c & a)}.

    \item \lstinline{take} (\ruleExistsI)
    
    Introduce un existencial instanciando su variable y reemplazándola por un
    término. Si la tesis es \lstinline{exists X . p(X)}, luego de
    \lstinline{take X := a}, se reduce a \lstinline{p(a)}.

    \item \lstinline{consider} (\ruleExistsE)
    
    Permite razonar sobre una variable que cumpla con un existencial. Si se
    puede justificar \lstinline{exists X. p(X)}, permite razonar sobre
    \lstinline{X}.

    El comando \lstinline{consider X st h: p by ...} agrega \lstinline{p} como
    hipótesis al contexto para el resto de la demostración. El \lstinline{by}
    debe justificar \lstinline{exists X. p(X)}.

    Valida que \lstinline{X} no esté libre en la tesis.

    También es posible usar una fórmula $\alpha$-equivalente, por
    ejemplo si podemos justificar \lstinline{exists X. p(X)}, podemos usarlo
    para \lstinline{consider Y st h: p(Y) by ...}

    \item \lstinline{let} (\ruleForallI)
    
    Permite demostrar un cuantificador universal. Si se tiene
    \lstinline{forall X. p(X)}, luego de \lstinline{let X} la tesis se reduce a
    \lstinline{p(X)} con un \lstinline{X} genérico. Puede ser el mismo nombre de
    variable o uno diferente, por ejemplo \lstinline{let Y}.

\end{itemize}

\subsection{Descarga de conjunciones}\label{ppa:sec:and-intro}

Si la tesis es una conjunción, se puede probar solo una parte de ella y se
reduce al resto.

\begin{figure}[H]
    \centering
    \caption{Descarga de conjunción simple}
    \begin{tabular}{c}
        \lstinputlisting{listings/interfaz/discharge-simple.ppa}
    \end{tabular}
\end{figure}

Esto puede ser prácticamente en cualquier orden.

\begin{figure}[H]
    \centering
    \caption{Descarga de conjunción compleja}
    \begin{tabular}{c}
    \lstinputlisting{listings/interfaz/discharge-complex.ppa}
    \end{tabular}
\end{figure}


\subsection{Otros comandos}

\begin{itemize}
    \item \lstinline{equivalently}: permite reducir la tesis a una fórmula
    equivalente. Útil para usar descarga de conjunciones.
    
    \lstinputlisting{listings/interfaz/equivalently.ppa}

    \item \lstinline{claim}: permite demostrar una afirmación auxiliar. Útil
    para ordenar las demostraciones sin tener que definir otro teorema. Ejemplo
    en \fullref{ppa:prog:alumnos}

    \begin{lstlisting}[numbers=none]
theorem t: <form1>
proof
    claim <name>: <form2>
    proof
        // Demostración de <form2>.
    end
    // Demostración de <form1> refiriéndose a <name>.
end
    \end{lstlisting}
\end{itemize}
