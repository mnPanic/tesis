PPA (\textit{Pani's Proof Assistant}) se construye sobre las fundaciones de
deducción natural. Es un proof assistant que permite escribir de una forma
práctica demostraciones de cualquier teoría de lógica clásica de primer órden.
Veamos un ejemplo

\todo{Ejemplo. Capaz puede ser uno que ya vimos en natural deduction para que sea más fácil la traducción, pero los interesantes tienen foralls}

\begin{itemize}
    \item Interfaz de PPA. Acá tienen que quedar claras todas las intuiciones
    desde el punto de vista de un usuario. Mencionar que es un buen momento para
    que vayan y prueben el programa (comando \texttt{check} nada más)
    \begin{itemize}
        \item Programas, teoremas, demostraciones como listas de pasos que
        reducen la tesis hasta agotarla.
        \item Comandos 1 por 1. Similar al README que ya existe pero más facha
        \item Ejemplos de demostraciones. Considerar incluir la de grupos
    \end{itemize}
    \item Compiladores
    \begin{itemize}
        \item Primer de compiladores en general y sus frontends
        \item Parser generators en general. LR/LALR
        \item Happy. Alex.
        \item Sintaxis EBNF. Incluir el archivo Alex/happy? Es cortito
    \end{itemize}
    \item Certificador: componente de PPA que "certifica" las demostraciones,
    generando un certificado en deducción natural. Implicó escribir muchos
    meta-teoremas.
    \begin{itemize}
        \item Formalización de muchos teoremas y axiomas: contextos (vale en el prefijo)
        \item Proof y proof steps, simplificación de la interfaz y mapeo de
        comandos a steps
        \item Implementación de cada comando
        \item By y solver para resolver varios. DNF. Extensión con foralls
        consecutivos. Demostración / justificación de que es correcto y completo
        para LP, pero heurístico para LPO (mostrar un caso en el que no funcione)
        \item Descarga de conjunciones
        \item Uso de dneg elim como razonamiento por el absurdo para demostrar
        deMorgan y equivalencias.
    \end{itemize}
\end{itemize}

\section{Interfaz}

\section{Compilador}

\section{Certificador}

\subsection{Unificación}
\label{ppa:sec:unification}