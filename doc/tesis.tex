\documentclass{report}

\usepackage[spanish]{babel}
\usepackage{amsmath}
\usepackage{amsthm}     % teoremas
\usepackage{float}      % [H]
\usepackage{abraces}    % \aunderbrace y \aoverbrace
\usepackage[dvipsnames]{xcolor}     % \textcolor
\usepackage[colorlinks=true]{hyperref}
\usepackage[spanish]{babel}

% https://tex.stackexchange.com/questions/32701/side-by-side-equations-with-equation-numbers-for-each
\usepackage{multicol}

% Teoremas, corolarios, etc.
% https://www.overleaf.com/learn/latex/theorems_and_proofs
\theoremstyle{definition} % Para que no salga en italicas

\newtheorem{theorem}{Teorema}
\newtheorem*{theorem*}{Teorema}

\newtheorem{lemma}{Lema}
\newtheorem*{lemma*}{Lema}

\newtheorem{prop}{Prop.}
\newtheorem*{prop*}{Prop}

\newtheorem{definition}{Def.}
\newtheorem*{definition*}{Def}

\newtheorem{obs}{Obs.}
\newtheorem*{obs*}{Obs}

\newtheorem{example}{Ejemplo}
\newtheorem*{example*}{Ejemplo}

% https://tex.stackexchange.com/questions/468/what-is-the-best-package-out-there-to-typeset-proof-trees
% https://mathweb.ucsd.edu/~sbuss/ResearchWeb/bussproofs/BussGuide2_Smith2012.pdf
% https://mathweb.ucsd.edu/~sbuss/ResearchWeb/bussproofs/
\usepackage{bussproofs}

\newcommand{\LL}[1]{\LeftLabel{#1}}
\newcommand{\RL}[1]{\RightLabel{#1}}

\newcommand{\ruleI}[1]{I#1}
\newcommand{\ruleE}[1]{E#1}

% Tipado
%% Contextos
\newcommand{\ctx}{\Gamma}
\newcommand{\ctxTwo}{\Delta}
\newcommand{\hypId}{h}
%% Fórmulas
\newcommand{\var}{x}
\newcommand{\term}{t}
\newcommand{\form}{A}
\newcommand{\formTwo}{B}
\newcommand{\formThree}{C}

%% Juicios
\newcommand{\judg}[2]{#1 \vdash #2}

% Misceláneos
\newcommand{\ubrace}[2]{\aunderbrace[l1r]{#1}_{#2}}
\newcommand{\obrace}[2]{\aoverbrace[L1R]{#1}^{#2}}

\newcommand{\todo}[1]{\textcolor{red}{(TODO: #1)}}

\title{Tesis de licenciatura}
\author{
    Manuel Panichelli
}
\date{\today}

\begin{document}
\maketitle

% TODO: Mover a .tex a parte y agregar con \include

Dos tipos de reglas

\begin{itemize}
    \item \textbf{Introducción}: Cómo demuestro?
    \item \textbf{Eliminación}: Cómo lo uso para demostrar?
\end{itemize}

\begin{multicols}{2}
    \begin{prooftree}
        \AxiomC{$\judg{\ctx}{\bot}$}
        \RL{\ruleE{$\bot$}}
        \UnaryInfC{$\judg{\ctx}{\form}$}
    \end{prooftree}
    \begin{prooftree}
        \AxiomC{}
        \RL{\ruleI{$\top$}}
        \UnaryInfC{$\judg{\ctx}{\top}$}
    \end{prooftree}
\end{multicols}

\begin{multicols}{2}
    \begin{prooftree}
        \AxiomC{}
        \RL{LEM}
        \UnaryInfC{$\judg{\ctx}{\form \vee \neg \form}$}
    \end{prooftree}
    \begin{prooftree}
        \AxiomC{}
        \RL{Ax}
        \UnaryInfC{$\judg{\ctx, \hypId:\form}{\hypId:\form}$}
    \end{prooftree}
\end{multicols}

\vspace*{0.5cm}


\begin{prooftree}
    \AxiomC{$\judg{\ctx}{\form}$}
    \AxiomC{$\judg{\ctx}{\formTwo}$}
    \RL{\ruleI{$\wedge$}}
    \BinaryInfC{$\judg{\ctx}{\form \wedge \formTwo}$}
\end{prooftree}

\begin{multicols}{2}
    \begin{prooftree}
        \AxiomC{$\judg{\ctx}{\form \wedge \formTwo}$}
        \RL{\ruleE{$\wedge_1$}}
        \UnaryInfC{$\judg{\ctx}{\form}$}
    \end{prooftree}
    \begin{prooftree}
        \AxiomC{$\judg{\ctx}{\form \wedge \formTwo}$}
        \RL{\ruleE{$\wedge_2$}}
        \UnaryInfC{$\judg{\ctx}{\formTwo}$}
    \end{prooftree}
\end{multicols}

\begin{multicols}{2}
    \begin{prooftree}
        \AxiomC{$\judg{\ctx}{\form}$}
        \RL{\ruleI{$\vee_1$}}
        \UnaryInfC{$\judg{\ctx}{\form \vee \formTwo}$}
    \end{prooftree}
    \begin{prooftree}
        \AxiomC{$\judg{\ctx}{\formTwo}$}
        \RL{\ruleI{$\vee_2$}}
        \UnaryInfC{$\judg{\ctx}{\form \vee \formTwo}$}
    \end{prooftree}
\end{multicols}

\begin{prooftree}
    \AxiomC{$\judg{\ctx}{\form \vee \formTwo}$}
    \AxiomC{$\judg{\ctx, \form}{\formThree}$}
    \AxiomC{$\judg{\ctx, \formTwo}{\formThree}$}
    \RL{\ruleE{$\vee$}}
    \TrinaryInfC{$\judg{\ctx}{\formThree}$}
\end{prooftree}

\ruleE{$\vee$} nos deja inferir una conclusión a partir de una disyunción dando sub demostraciones que muestran como la conclusión se puede deducir asumiendo cualquiera de los elementos.

\begin{prooftree}
    \AxiomC{$\judg{\ctx, \form}{\formTwo}$}
    \RL{\ruleI{$\to$}}
    \UnaryInfC{$\judg{\ctx}{\form \to \formTwo}$}
\end{prooftree}

\begin{prooftree}
    \AxiomC{$\judg{\ctx}{\form \to \formTwo}$}
    \AxiomC{$\judg{\ctx}{\form}$}
    \RL{\ruleE{$\to$} \scriptsize (modus ponens)}
    \BinaryInfC{$\judg{\ctx}{\formTwo}$}
\end{prooftree}

\vspace{0.5cm}

\begin{multicols}{2}
    \begin{prooftree}
        \AxiomC{$\judg{\ctx, \form}{\bot}$}
        \RL{\ruleI{$\neg$}}
        \UnaryInfC{$\judg{\ctx}{\neg \form}$}
    \end{prooftree}
    
    \begin{prooftree}
        \AxiomC{$\judg{\ctx}{\neg \form}$}
        \AxiomC{$\judg{\ctx}{\form}$}
        \RL{\ruleE{$\neg$}}
        \BinaryInfC{$\judg{\ctx}{\bot}$}
    \end{prooftree}
    
\end{multicols}

\todo{Validar las justificaciones coloquiales de acá}

Las reglas de $\forall$ y $\exists$ se pueden ver como extensiones a las de $\wedge$ y $\vee$.

Un $\forall$ se puede pensar como una conjunción con un elemento por cada uno dl dominio sobre el cual se cuantifica.

\begin{multicols}{2}
    \begin{prooftree}
        \AxiomC{$\judg{\ctx}{\form}$}
        \AxiomC{$x \notin fv(\ctx)$}
        \RL{\ruleI{$\forall$}}
        \BinaryInfC{$\judg{\ctx}{\forall \var.\form}$}
    \end{prooftree}
    \begin{prooftree}
        \AxiomC{$\judg{\ctx}{\forall \var.\form}$}
        \RL{\ruleE{$\forall$}}
        \UnaryInfC{$\judg{\ctx}{\form \{\var := \term\}}$}
    \end{prooftree}
\end{multicols}

\begin{itemize}
    \item \ruleE{$\forall$}: Para usar un $\forall x.\form$ para demostrar (eliminar) instancio el $x$ en cualquier \textit{término} $t$, ya que es válido para todos.
    \item \ruleI{$\forall$}: Para demostrar (introducir) un $\forall x. \form$, quiero ver que sin importar el valor que tome $x$ yo puedo demostrar $\form$. Pero para eso en mi contexto $\Gamma$ no tengo que tenerlo ligado a nada, sino no lo estaría demostrando en general
\end{itemize}

\begin{prooftree}
    \AxiomC{$\judg{\ctx}{\form\{\var := \term\}}$}
    \RL{\ruleI{$\exists$}}
    \UnaryInfC{$\judg{\ctx}{\exists \var. \form}$}
\end{prooftree}

\begin{prooftree}
    \AxiomC{$\judg{\ctx}{\exists \var.\form}$}
    \AxiomC{$\judg{\ctx, \form}{\formTwo}$}
    \AxiomC{$x \notin fv(\ctx, \formTwo)$}
    \RL{\ruleE{$\exists$}}
    \TrinaryInfC{$\judg{\ctx}{\formTwo}$}
\end{prooftree}


\begin{itemize}
    \item \ruleI{$\exists$}: Para demostrar un $\exists$, alcanza con instanciar la variable en un término $t$ que sea válido.
    \item \ruleE{$\exists$}: Para usar un $\exists$ para demostrar, es parecido a \ruleE{$\vee$}. Como tenemos que ver que vale para cualquier $\var$, podemos concluir $\formTwo$ tomando como hipótesis $\form$ con $\var$ sin instanciar. 
\end{itemize}

\end{document}